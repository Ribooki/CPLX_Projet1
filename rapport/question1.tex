\documentclass[a4paper,12pt]{article}

\usepackage{mathptmx}
\usepackage[utf8]{inputenc}
\usepackage[french]{babel}
\frenchbsetup{StandardLists=true}
\usepackage[T1]{fontenc}
%\usepackage[left=2cm,right=2cm,top=2cm,bottom=2cm]{geometry}
\usepackage{hyperref}
\usepackage{enumerate}
\usepackage{pifont}
\usepackage{wrapfig}
\usepackage{multicol}
\usepackage{verbatim}
\usepackage{xcolor}
\usepackage{sectsty}
\usepackage{amsthm}
\usepackage{amsmath,amsfonts,amssymb}
\usepackage{mathrsfs}

%Packages Figures et graphiques
\usepackage{graphics} %inclusion de figures
\usepackage{graphicx} %inclusion de figures
\usepackage{pstricks,pst-node} %Graphiques

%pour écrire des algo
\usepackage{algorithm}
\usepackage{algorithmic}
%\usepackage[]{algorithm2e}

\partfont{\centering}
%\sectionfont{\color{red}{}}

\newcommand{\reels}{\mathbb{R}}
\newcommand{\naturels}{\mathbb{N}}

\renewcommand{\algorithmicrequire}{\textbf{Entrée:}}
\renewcommand{\algorithmicensure}{\textbf{Sortie:}}
\renewcommand{\algorithmicwhile}{\textbf{tant que}}
\renewcommand{\algorithmicdo}{\textbf{faire}}
\renewcommand{\algorithmicendwhile}{\textbf{fin tant que}}
\renewcommand{\algorithmicif}{\textbf{si}}
\renewcommand{\algorithmicendif}{\textbf{fin si}}
\renewcommand{\algorithmicthen}{\textbf{alors}}
\renewcommand{\algorithmicreturn}{\textbf{retourner}}

\begin{document}
 
\part*{Pseudo-code de la question 1}
 
\begin{algorithm*}
    \caption{arc}
    \begin{algorithmic}
        \REQUIRE graphe g, sommet x, sommet y
        \ENSURE 1 s'il existe un arc de x à y, et 0 sinon
        
        \WHILE{on n'a pas fini d'explorer tous les arcs de x}
            \IF{il existe un arc de x à y}
                \RETURN 1
            \ENDIF
        \ENDWHILE
        \RETURN 0
    \end{algorithmic}
\end{algorithm*}
 
\begin{algorithm*}
    \caption{isDesert}
    \begin{algorithmic}
        \REQUIRE graphe g, ens\_de\_sommets e
        \ENSURE 1 si le graphe est désert, et 0 sinon
        
        \WHILE{on n'a pas fini d'explorer tous les sommets de e}
            \IF{arc(g, sommet actuel, sommet suivant)}
                \RETURN 0
            \ENDIF
            
            \IF{arc(g, sommet suivant, sommet actuel)}
                \RETURN 0
            \ENDIF
        \ENDWHILE
        
        \RETURN 1
    \end{algorithmic}
\end{algorithm*}

 
\end{document}
